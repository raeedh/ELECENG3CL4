\documentclass[12pt]{article}
\usepackage[letterpaper, margin=1in]{geometry}
\usepackage{graphicx}
\usepackage{subcaption}
\graphicspath{{./Figures/}}
\usepackage{hyperref}
\usepackage{parskip}
\usepackage{amsmath}
\usepackage{amssymb}
\usepackage{mathrsfs}
\usepackage{enumitem}
\allowdisplaybreaks

\title{ELECENG 3CL4 Lab 2 Report}
\author{
    Aaron Pinto \\
    pintoa9 \\
    L02
    \and
    Raeed Hassan \\
    hassam41 \\
    L02
}

\begin{document}

\maketitle
\clearpage

\setcounter{section}{2}
\section{Perform Closed Loop Identification}

\subsection{Experiment 1: Time Domain Identification}
% iii) ran sim
% iv) K = 2
% v) measure height of first overshoot peak, time of first overshoot peak
% vi) measure the stuff
% vii) measure time difference from edge to time of first peak
% viii) record measurements
% ix) use measurements to calculate motor parameters

\subsection{Experiment 2: Frequency Domain Identification}
% vi) observe at 0.5 hz; small delay, some non-linear effects at direction change
% vii) 1 hz; slightly greater gain, slight longer delay, non-linear effects diminished
% viii) 2 hz; gain signif greater, moderate delay, non-linear effects negligible
% ix) 3 hz; gain signif greater, signifc delay, non-linear effects negligible
% x) 4 hz; gain smaller than 3 hz, output almost completely out of phase w/ input
% xi) 5 hz; 
% xii) 6 hz;
% xiii) search for freq where mag reaches peak
% xiv) use measurements to calculate motor parameters

\end{document}