\documentclass[12pt]{article}
\usepackage[letterpaper, margin=1in]{geometry}
\usepackage{graphicx}
\usepackage{subcaption}
\graphicspath{{./Figures/}}
\usepackage{hyperref}
\usepackage{parskip}
\usepackage{amsmath}
\usepackage{amssymb}
\usepackage{mathrsfs}
\usepackage{enumitem}
\usepackage{gensymb}
% \allowdisplaybreaks

\title{ELECENG 3CL4 Lab 5 Pre-lab}
\author{
    Aaron Pinto \\
    pintoa9 \\
    L02
    \and
    Raeed Hassan \\
    hassam41 \\
    L02
}

\begin{document}

\maketitle
\clearpage

\begin{enumerate}
	\item %1

	\item %2
	
	\item %3
	
	\item %4
	
	\item %5
	$K_c = 62.1$, $z_{lead} = 6$, $p_{lead} = 15.51$
	
	\item %6
	
	\item %7
	poles at $-6 \pm 15j$

	\item %8
	
	\item %9
	We can choose values of $z_{lag}$ and $p_{lag}$ that are near the origin. This would not affect the transient performance of the closed loop very much as the transient responses of the zero and pole would decay very quickly. 
	
	\item %10
	
\end{enumerate}

\end{document}
